%
\documentclass[11pt]{article}

%\usepackage{times}
%\usepackage[margin=.75in]{geometry}
\usepackage[margin=1in]{geometry}
\usepackage[dvipsnames,usenames]{xcolor}
% \usepackage[numbers]{natbib}


\begin{document}


\begin{center}
  {\Large\bf The Chinese University of Hong Kong \\
  \ \\
ECON5170  Computational Methods in Economics \\
Spring, 2017-2018}
\end{center}

 \vspace{2pt}

\begin{description}
\item[Instructors] Naijia Guo (guonaijia@cuhk.edu.hk, ELB 913) \\
  Zhentao Shi (zhentao.shi@cuhk.edu.hk, ELB 912)
\end{description}

\section{Lecture Hours and Location}
Time: January 10th - April 18th, every Wednesday 14:30 - 17:15, except public holidays \\
Venue: Wong Foo Yuan Building (FYB) 107A\\
Office hours: By appointment




\section{Course Description}

In modern economic research, computers enhance our capacity of solving complex problems. Computation is particularly important in fields involving massive data. The objective of this course is to introduce graduate students to computational approaches for solving economic models, with an emphasis on dynamic programming and simulation-based econometric methods. We will formulate economic problems in computationally tractable form and use techniques from numerical analysis to solve them. The substantive applications will cover a wide range of problems including labor, industrial organization, macroeconomics, and international trade.

\section{Learning outcomes}

Computational economics has not been part of the core curriculum of postgraduate-level economics education, whereas programming skill is critical for a postgraduates success in academia and industry. This course intends to teach students computational methods for solving economic problems, and expose students to extensive programming exercises. We expect that at the end of the course a student would proficiently use at least one programming language (Stata, Matlab, R, etc). Moreover, we aim to equip the students with the computational ability to tackle problems of their own research areas.

\section{Assessment}
\begin{tabular}{p{0.5in}p{0.5in}p{5in}}
   Midterm & 30\% & A small take-home exercise. \\
   Final  & 70\%  & A group project. Form a group of 2-3 people. Write a computer program to solve one of the three problems (micro, macro, or metrics). Present the results on April 18th or later (TBA). Hand in the final codes by May 6th.
\end{tabular}

\section{Class Schedule}
\begin{tabular}{p{1in}p{4in}}
  \hline
  Date & Content \\
  \hline
  10 Jan & Basic R \\
  17 Jan & Advanced R \\
  24 Jan & Basic Stata (in Undergraduate Computer Lab ELB 916) \\
  31 Jan & Advanced Stata (in Undergraduate Computer Lab ELB 916)  \\
  7 Feb &  Monte Carlo Simulation\\
  14 Feb & Numerical Integration \\
  28 Feb & Numerical Optimization  \\
  7 Mar &  Machine Learning  \\
  14 Mar & Linear Equations \\
  21 Mar & Nonlinear Equations \\
  28 Mar & Approximation methods \\
  4 Apr & Dynamic programming \\
  (TBA) & Presentation of group projects \\
  \hline
\end{tabular}





\section{Required Readings}
Judd, Kenneth (1998): Numerical Methods in Economics, the MIT Press \\
Efron and Hastie (2016): Computer Age Statistical Inference: Algorithms, Evidence, and Data Science, Cambridge University Press


% \nocite*{}
% \bibliographystyle{econometrica}
\bibliographystyle{apa}
\bibliography{book_version/book}



\end{document}






