\documentclass[11pt]{article}

%\usepackage{times}
%\usepackage[margin=.75in]{geometry}
\usepackage[margin=1in]{geometry}
\usepackage[dvipsnames,usenames]{xcolor}
\usepackage[letterpaper=true,colorlinks=true,pdfpagemode=none,urlcolor=Blue,linkcolor=RoyalBlue,citecolor=OliveGreen,pdfstartview=FitH,linktoc=page]{hyperref}
\usepackage{url}
\urlstyle{same}


\begin{document}


\begin{center}
  {\Large\bf The Chinese University of Hong Kong \\
  \ \\
ECON5170  Computational Methods in Economics \\
Spring, 2018-2019}
\end{center}

 \vspace{2pt}

\begin{description}
\item[Instructors] Naijia Guo (guonaijia@cuhk.edu.hk, ELB 913) \\
  Zhentao Shi (zhentao.shi@cuhk.edu.hk, ELB 912)
\end{description}

\section{Lecture Hours and Location}
Time: January 10th - April 18th, every Thursday 9:30 - 12:15 \\
Venue: Esther Lee Bldg 304 \\
Office hours: By appointment




\section{Course Description}

In modern economic research, computers enhance our capacity of solving complex problems. Computation is particularly important in fields involving massive data. The objective of this course is to introduce graduate students to computational approaches for solving economic models, with an emphasis on dynamic programming and simulation-based econometric methods. We will formulate economic problems in computationally tractable form and use techniques from numerical analysis to solve them. The substantive applications will cover a wide range of problems including labor, industrial organization, macroeconomics, and international trade.

\section{Learning outcomes}

Computational economics has not been part of the core curriculum of postgraduate-level economics education, whereas programming skill is critical for a postgraduates success in academia and industry. This course intends to teach students computational methods for solving economic problems, and expose students to extensive programming exercises. We expect that at the end of the course a student would proficiently use at least one programming language (Stata, Matlab, R, etc). Moreover, we aim to equip the students with the computational ability to tackle problems of their own research areas.

\section{Assessment}
\begin{tabular}{p{0.5in}p{0.5in}p{5in}}
   Midterm & 30\% & A small take-home exercise. \\
   Final  & 70\%  & A group project. Form a group of 2-3 people. Write a computer program to solve one of the three problems (micro, macro, or metrics). Present the results on April 18th or later (TBA). Hand in the final codes by May 6th.
\end{tabular}

\section{Class Schedule}
\begin{tabular}{p{1in}p{4in}}
  \hline
  Date & Content \\
  \hline
  10 Jan & Basic R \\
  17 Jan & Advanced R \\
  24 Jan & Numerical Integration  \\
  31 Jan & Numerical Optimization  \\
  7 Feb & No class (Lunar New Year)\\
  14 Feb & Basic Stata (in Undergraduate Computer Lab ELB 916) \\
  21 Feb & Advanced Stata (in Undergraduate Computer Lab ELB 916)\\
  28 Feb & Machine Learning I  \\
  7 Mar &  Machine Learning II \\
  14 Mar & Linear Equations \\
  21 Mar & Nonlinear Equations \\
  28 Mar & Approximation methods \\
  4 Apr & Dynamic programming \\
  11 Apr & Office hour \\
  18 Apr (TBA) & Presentation of group projects \\   
  \hline
\end{tabular}


\section{Required Readings}
Judd, Kenneth (1998): Numerical Methods in Economics, the MIT Press \\
Efron and Hastie (2016): Computer Age Statistical Inference: Algorithms, Evidence, and Data Science, Cambridge University Press


\section{Recommended Readings}
Altonji, J. G., \& Segal, L. M. (1996). Small-sample bias in GMM estimation of covariance structures. Journal of Business and Economic Statistics, 14(3), 353-366. \\
Andrews, D. W. (2000). Inconsistency of the bootstrap when a parameter is on the boundary of the parameter space. Econometrica, 68(2), 399-405. \\
Armstrong, T., Bertanha, M. \& Hong, H.(2014) A Fast Bootstrap Method for Parametric and Semi-parametric Model, the Journal of Econometrics.179(2), 128-133. \\
Bajari, P., Benkard, C. L., \& Levin, J. (2007). Estimating dynamic models of imperfect competition. Econometrica, 75(5), 1331-1370. \\
Chernozhukov, V., \& Hong, H. (2003). An MCMC approach to classical estimation. Journal of Econometrics, 115(2), 293-346. \\
% Efron, B., \& Tibshirani, R. J. (1994). An introduction to the bootstrap (Vol. 57). CRC press. \\
Hansen, L. P., Heaton, J., \& Yaron, A. (1996). Finite-sample properties of some alternative GMM estimators. Journal of Business and Economic Statistics,14(3), 262-280. \\
Koenker, R. (2005). Quantile regression (No. 38). Cambridge university press. \\
Tibshirani, R. (1996) Regression shrinkage and selection via the lasso. Journal of the Royal Statistical Society. Series B, 267-288. \\
Li, Q., \& Racine, J. S. (2007). Nonparametric econometrics: Theory and practice. Princeton University Press. \\
% Owen, A. B. (2010). Empirical likelihood. CRC press. \\
Pakes, A., \& Pollard, D. (1989). Simulation and the asymptotics of optimization estimators. Econometrica, 1027-1057. \\
Shi, Z., (2016). Econometric Estimation with High-Dimensional Moment Equalities. Journal of Econometrics, 195, 104-119 \\
Su, C. L., \& Judd, K. L. (2012). Constrained optimization approaches to estimation of structural models. Econometrica, 80(5), 2213-2230.\\
Su, L., Shi, Z., and \& Phillips, P. C. B. (2016). Identifying Latent Structures in Panel Data. Econometrica, 84(6), 2215-2264
\end{document}






